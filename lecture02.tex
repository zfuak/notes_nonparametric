\begin{exercise}
    Let $\calh$ be a $\sigma$-algebra such that all elements of $\calh$  belong to $\sigma(X)$. We can show that \begin{equation*}
        \E\bra{\E\bra{Y|X}|\calh} = \E\bra{Y|\calh}
    \end{equation*}
    We can think of this in terms of projections. The projection of $Y$ onto $\calh$ is $\E\bra{Y|\calh}$, and the projection of $\E\bra{Y|X}$ onto $\calh$ is $\E\bra{\E\bra{Y|X}|\calh}$. The equality says that the projection of the projection of $Y$ onto $\calh$ is the same as the projection of $Y$ onto $\calh$.
\end{exercise}
\subsection{Identification}
We are given data consisting of draws from a distribution law $\p_{Y,X}$ where $Y,X$ are observable vectors.
An economic model consists of 
\begin{enumerate}
    \item An equation $v(Y,\gamma,X,\varepsilon;\zeta)=0$ where $v$ is a system of functions, $\gamma$ is a vector of variables that is determined within the model but unobservable, $\varepsilon$ is a vector of variables that is determined within the model and unobservable. $\zeta$ is a vector of functions and distributions.
    \item Restrictions: $\zeta \in \calr$ where $\calr$ is a set of restrictions.
\end{enumerate}
For any $\zeta \in \calr$, $\p_{Y,X;\zeta}$ is the distribution law of the observables generated by $\zeta$. We denote the true structural parameter by $\zeta^*$. \textbf{We oftern care about $\psi^*=\Psi(\zeta^*)$ where $\Psi$ is a mapping from $\calr$ to $\calp$ and $\calp$ is the parameter space.}
We define the identified set as
\begin{equation*}
    \Gamma_{Y,X}(\psi,\calr) = \set{\p_{Y,X;\zeta}:\zeta \in \calr \quad \text{s.t. }\Psi(\zeta)=\psi}.
\end{equation*} It is the set of all distributions of the observables that are consistent with the model and the restrictions, that is, generated by $\zeta$ contained within the restriction. 
\begin{definition}[Identification]
    We say that $\psi^*$ is identified if for any $\psi^* \in \calp$ if $ \Gamma_{Y,X}(\psi^*,\calr) \cap \Gamma_{Y,X}(\psi,\calr) \neq \emptyset$, then $\psi^*=\psi$.
\end{definition}
\begin{exercise}
    We specify a linear model $Y=f(X)+\varepsilon$ where $f$ is continuous near $x_0 \in \text{supp}(X)$, and $\E\bra{\abs{\varepsilon}+\abs{f(X)}}<\infty$ and $\E\bra{\varepsilon|X}=0$. We can show that $\psi^*=\E\bra{f(X)}$ is identified because under these conditions $f(X)=\E\bra{f(X)}$, the conditional expectation. The system of equations is $v(Y,\gamma,X,\varepsilon;\zeta)=Y-f(X)-\varepsilon=0$. The restriction is $\zeta = (f.\p_{X,\varepsilon})\in \calr$. 
\end{exercise}
\begin{proof}
     Assume that there are two $zeta$ that satisfy the restrictions and generate the same distribution of the observables. $$(f,\p_{X,\varepsilon}), (f^*,\p_{X,\varepsilon}^*) \overbrace{\to}_{generate} \p_{Y,X}$$
     Then we have the following \begin{itemize}
        \item $\int \p_{Y,X(y,x)}(y,\cdot)dy=\int \p_{\varepsilon,X}(e\cdot)de=\int \p^*_{\varepsilon,X}(e,\cdot)=\p_X(x)$
        \item $\E\bra{Y|X}=f(X)=f^*(X)$
        \item Because $f$ is identified, $\p_{\varepsilon,X}=\p_{Y-f(X),X}=\p_{Y-f^*(X),X}=\p_{\varepsilon,X}^*$
     \end{itemize}
     
\end{proof}
Now that we have introduced the basic nonparametric model, we introduce nonparametric model with instrumental variables, where $\E\bra{\varepsilon|Z}=0$. The identification requires an additional restriction -- \emph{Completeness}. 
\begin{definition}[Completeness]
    For any $\phi$ such that $\E\bra{\abs{\phi(X)}}<\infty$, $\E\bra{\phi(X)\mid Z}=0$ implies that $\phi(x)=0$ on the support of $X$.
\end{definition}
\paragraph{Discrete case}
When $(X,Z)$ is discrete finite, $$\text{supp}(X)=\set{x_1,\dots,x_n}, \text{supp}(Z)=\set{z_1,\dots,z_m}$$
We can write the completeness condition as \begin{equation*}
    E\bra{\phi(X)\mid Z=z_j}= \sum \phi(x_i)\p_{X|Z}(X=x_i|Z=z_j)=0 \quad \forall j
\end{equation*}
This is a system of $m$ equations in $n$ unknowns. We can show that the completeness condition is satisfied if and only if $m\ge n$. 
In the following section, we discuss a specific continuous case ($\text{supp}(X,Z)\in \R^2$) where $X=Z-\eta$ and $Z$ is independent of $\varepsilon$. Both have densities and  $\eta\sim \calu\pa{[-1,1]}$.
    