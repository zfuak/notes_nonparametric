%%%

\subsection{Symmetric kernel}
\begin{theorem}[Symmetric kernel]\label{thm:sym_ker}
	Let $f_X\in L^2\pa{\R}, K\in L^2\pa{\R}$ be a symmetric kernel such that
	\begin{equation*}\label{eq:tocheck}
		\sup_{w\in \R\setminus \set{0}} \frac{\abs{1- \F \bra{K} \pa{w}}}{\abs{w}^{\beta'}}\leq A <\infty
	\end{equation*}
	for some $\beta',A>0$. Then
	\begin{equation*}
		\sup_{f_X\in \calp_S \pa{\beta, L}} \E\bra{\norm{\hat{f}_X - f_X}^2_2}\leq C n^{-\frac{2\tilde{\beta}}{2\tilde{\beta}+1}},
	\end{equation*}
	where $\tilde{\beta}=\min \set{\beta,\beta'}$, if $h= \alpha n^{-\frac{1}{2\tilde{\beta} +1}}$ for some $\alpha >0$ and $C$ is a constant which only depends on $L,\alpha,A,K$.
\end{theorem}

\begin{example}\label{exa:fourier_kernel}
	\begin{enumerate}
		\item Gaussian kernel: $K\pa{u} = \frac{1}{\sqrt{2\pi}} e^{-\frac{u^2}{2}}$,
		      $\F\bra{K}\pa{u} = e^{-\frac{u^2}{2}}$. We have
		      \begin{equation*}
			      \frac{\abs{1-e^{-w^2 /2 }}}{\abs{w}^{\beta'}} \leq \begin{cases} \abs{w}^{-\beta'},              & \abs{w}\geq 1 \\
              \frac{w^2/2}{\abs{w}^{\beta'}}, & \abs{w}\leq 1\end{cases}
		      \end{equation*}
		      so \ref{eq:tocheck} holds if $\beta'\leq 2$, else the $\sup$ is $\infty$.
		\item The sinc kernel: $K\pa{u} = \frac{\sin\pa{u}}{\pi u}, \F\bra{K}\pa{w}=
			      \one_{\abs{w}\leq 1}$. We have
		      \begin{equation*}
			      \frac{\abs{1- \F\bra{K}\pa{w}}}{\abs{w}^{\beta'}} \leq \begin{cases}
				      \abs{w}^{-\beta'}, & \abs{u}> 1     \\
				      0,                 & \abs{u}\leq 1,
			      \end{cases}
		      \end{equation*}
		      so \ref{eq:tocheck} holds for all $\beta'$. Such a kernel is called an \textbf{infinite power kernel} or \textbf{superkernel}.
		\item Trapeze kernel: Let \begin{equation*}
			      \F\bra{K}\pa{w} = \begin{cases}
				      0,             & \abs{w} >a        \\
				      1,             & \abs{w}\leq b     \\
				      \text{linear}, & \text{otherwise},
			      \end{cases}
		      \end{equation*}
		      a trapeze. Then \ref{eq:tocheck} holds for all $\beta'$.
		      Let us write $K_2$ for the trapeze (in Fourier space) and $K_1$ for the sinc Kernel (see \ref{exa:fourier_kernel}). Then
		      \begin{equation*}
			      K_2 = \frac{1}{2\pi} \F\bra{\F\bra{K_1}\ast F\bra{K_1}} = \frac{1}{2\pi} \F\bra{\F\bra{K_1}}\F\bra{\F\bra{K_1}} = 2\pi K_1^2\pa{u} = 2\pi \pa{\frac{\sin u}{\pi u}},
		      \end{equation*}
		      which is in $L^1\pa{\R}\cap L^2\pa{\R}$.
	\end{enumerate}
\end{example}
\paragraph{Optimal rate of convergence} It can be shown that the \emph{sinc} kernel has the optimal rate of
convergence.

A Corollary of the Theorem \ref{thm:sym_ker} that we have seen for
cross-validation is
\begin{corollary}
	Let $K$ be the sinc kernel, then
	\begin{equation*}
		\sup_{f_X\in \calp_S\pa{\beta,L}}\E\bra{\norm{\hat{f}_X^{\CV} - f_X}_2^2 }\leq Cn^{-\frac{2\beta}{2\beta +1}}
	\end{equation*}
	for all $\beta > \frac{1}{2}, L>0$, where $C$ only depends on $\beta$ and $L$.
\end{corollary}
Some people have shown:
\begin{proposition}
	\begin{equation*}
		\inf_{\hat{f}}\sup_{f_X\in \calp_S\pa{\beta,L}}\E\bra{\norm{\hat{f}_X - f_X}^2_2}\geq C_\ast n^{-\frac{2\beta}{2\beta +1}}
	\end{equation*}
	for some absolute constant $C_\ast$.
\end{proposition}

This means that $n^{-\frac{2\beta}{2\beta +1 }}$ is the ``minimax'' optimal
rate of convergence and the cross-validated estimator is minimax adaptive
(i.e.~we can construct it with the data only).

\paragraph{Kernel comparison} We end this section by the following table.
\begin{table}[!h]
	\centering
	\begin{tabular}{l|c|c|c}
		name         & kernel & $\F\bra{K}$ & \ $\frac{\abs{1- \F \bra{K} \pa{w}}}{\abs{w}^{\beta}}$ \\
		\hline
		Gaussian     &        &             &                                                        \\
		Epanechnikov &        &             &                                                        \\
		Sinc         &        &             &                                                        \\
		Trapeze      &        &             &                                                        \\
	\end{tabular}
	\caption{Summary}
\end{table}

%%%

