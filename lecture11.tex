\section{Treatment Effects}
\subsection{Setup} We have a dataset $\bra{Y_i,D_i,X_i,Z_i,W_i}_{i=1}^n$ following i.i.d. from a
joint distribution.
\begin{itemize}
    \item $D$ is a binary treatment variable, $D\in\bra{0,1}$.
    \item $Y$ is the outcome variable. Here $Y$ is a random variable $Y\in \R$.
    \item $X,Z,W$ are covariates/additional random variables.
\end{itemize}
The model equation (for each individual $i$) is $y_i=y\pa{0}(1-D)+y\pa{1}D$. The potential outcome is $y_i\pa{1},y_i\pa{0}$, which are not observed. We can only observe $y_i=y\pa{D_i}$.

\subsection{Parameters of Interest}
\begin{itemize}
    \item Average treatment effect (ATE): $\tau = \E\bra{Y\pa{1}-Y\pa{0}}$
    \item Conditional average treatment effect (CATE): $\tau(x) =
              \E\bra{Y\pa{1}-Y\pa{0}|X=x}$. It can be useful if we care about the effect of
          the treatment on a specific subgroup of the population.
    \item Average treatment effect on the treated (ATT): $\tau_{\text{ATT}} =
              \E\bra{Y\pa{1}-Y\pa{0}|D=1}$
    \item Average treatment effect on the untreated (ATU): $\tau_{\text{ATU}} =
              \E\bra{Y\pa{1}-Y\pa{0}|D=0}$
    \item Conditional average treatment effect on the treated (CATT):
          $\tau_{\text{ATT}}(x) = \E\bra{Y\pa{1}-Y\pa{0}|D=1,X=x}$
\end{itemize}

\subsection{Identification}
We need to impose some assumptions in order to identify the parameters.
\begin{enumerate}
    \item $\p\pa{D=1}\in \pa{0,1}$
    \item The covariates $X,Z,W$ are such that if $X=X(0)+D\pa{X(1)-X(0)}$, then
          $X(1)=X(0)$.
    \item The potential outcome $Y\pa{1},Y\pa{0}$ are independent of $D$ which is
          $Y\pa{1},Y\pa{0}\indep D$
    \item The potential outcome $Y\pa{1},Y\pa{0}$ and $X$ are independent of $D$, that is
          $$Y\pa{1},Y\pa{0},X \indep D$$
\end{enumerate}
We introduce a new notation for the purpose of another assumption.
\begin{definition}[Propensity score]
    The propensity score is defined as the conditional probability of receiving the treatment given the covariates, that is \[\pi(x)=\p(D=1\mid X=x)\]
\end{definition}
\begin{remark}
    Later we will build estimators using propensity score, called \textbf{inverse propensity score weighting (IPSW) estimator}.
\end{remark}
\paragraph{Common support}
The propensity score $\pi(x)$ continuous and bounded between 0 and 1 for all
$x\in \text{supp}(X)$ \\

Sometimes, treatment $D$ is assigned randomly conditional on $X$. We introduce
the following assumptions
\paragraph{Unconfoundedness}
The treatment $D$ is unconfounded with the potential outcome $Y$ given $X$ if
\begin{equation*}
    Y\pa{1},Y\pa{0}\indep D\mid X
\end{equation*}
\paragraph{Conditional mean independence}
The potential outcome $Y\pa{1},Y\pa{0}$ are independent of $D$ given $X,Z,W$,
that is \[\E\bra{Y(d)\mid D,X}=\E\bra{Y(d)\mid X}\] Later we will show that unconfoundedness implies conditional mean independence.