\section{MISE and Cross validation}
\subsection{MISE}
To define the $\MISE$, we would like
\begin{equation*}
	\E\bra{\norm{\hat{f}_X - f_X}^2_2} < \infty.
\end{equation*}
We assume $f_X\in L^2\pa{\R}$. We would like as well $\hat{f}_X\in L^2\pa{\R}$. This is true if $K\in L^2\pa{\R}$.

Indeed,
\begin{equation*}
	\begin{split}
		\norm{\hat{f}_X}_2^2 & \leq \frac{2^{n-1}}{\pa{nh}^2} \sum_{i=1}^n \int_\R K\pa{\frac{X_i -x}{h}}^2 dx \\
		                     & \leq \frac{2^{n-1}}{nh} \int_{\R} K^2\pa{u} du <\infty.
	\end{split}
\end{equation*}
The idea behind this inequality is $\pa{a+b}^2 \leq 2 \pa{a^2 + b^2}$, and then by induction, $\pa{\sum_{i=1}^n a_i }^2 \leq 2^{n-1} \sum_{i=1}^n a_i^2$.
\subsection{Cross validation}
Let us write
\begin{equation*}
	\begin{split}
		\mathrm{MISE}\pa{h} & = \E\bra{\int_\R \pa{\hat{f}^h_X\pa{x}-f_X\pa{x} }^2dx}                                                                                  \\
		                    & = \underbrace{\E\bra{\int_\R \pa{\hat{f}^h_X \pa{x}}^2 dx - 2\int_\R \hat{f}^h_X\pa{x} f_X\pa{x}dx }}_{=I\pa{h}} + \int_\R f_X^2\pa{x}dx
	\end{split}
\end{equation*}
% \cite{rudemo} 
introduced
\begin{equation*}
	\widehat{\mathrm{CV}}\pa{h} = \int_\R \hat{f}_X^2 \pa{x}dx - \underbrace{\frac{2}{n} \sum_{i=1}^n \hat{f}_{X,-i}\pa{X_i}}_{=\hat{A}},
\end{equation*}
where $\hat{f}_{X,-i}\pa{x} = \frac{1}{\pa{n-1}h} \sum_{j=1,j\neq i}^n K\pa{\frac{X_j-x}{h}}$. The cross-validated bandwidth is
\begin{equation*}
	\hat{h}_{\mathrm{CV}} = \argmin_{h>0 }\widehat{\mathrm{CV}}\pa{h}
\end{equation*}
We claim
\begin{equation*}
	\frac{1}{2}\E\bra{\hat{A}}= \E\bra{\int_\R \hat{f}_X\pa{x} f_X\pa{x} dx}
\end{equation*}
We have
\begin{equation*}
	\begin{split}
		\E\bra{\hat{f}_{X,-1}\pa{X_1}} & = \E_{\p_{X_2}\otimes  \cdot\otimes \p_{X_n}}\bra{\int_\R \hat{f}_{X,-i}\pa{x}f_X\pa{x} dx} \\
		                               & = \E\bra{\frac{1}{\pa{n-1} h}\sum_{j=2}^n \int_\R K\pa{\frac{X_j-x}{h}}f_X\pa{x} dx}        \\
		                               & =\frac{1}{h} \int_\R\int_\R K\pa{\frac{z-x}{h}} f_X\pa{z} f_X\pa{x} dz dx
	\end{split}
\end{equation*}
As an exercise, show that this yields the claim.

\begin{theorem}[Oracle inequality]
	\label{thm:dalelane}
	Let $f_{\max}$ be such that for all $x$, $f_X\pa{x}\leq f_{\max} <\infty$. Assume the kernel $K$ is such that
	$\int_\R K^2\pa{u}du<\infty$. $\F\bra{K}\geq 0$ and $\mathrm{supp}\pa{\F\bra{K}}\subseteq [-1,1]$. Then $\hat{f}_X^\ast = \hat{f}_X^{h_{\mathrm{CV}}}$ is such that for all $0<\delta <1$, for all $n\geq 1$,
	\begin{equation*}
		\E\bra{\int_\R \pa{\hat{f}^\ast_X \pa{x} - f_X\pa{x}}^2dx} \leq \pa{1+\frac{C}{n^\delta}} \min_{h>\frac{1}{n}} \E\bra{\int \pa{\hat{f}_X^h\pa{x} - f_X\pa{x}}^2 dx} + \frac{C\pa{\log n}^{\frac{\delta}{2}}}{n^{1-\delta}}
	\end{equation*}
\end{theorem}
\begin{remark}
	The cross-validation bandwidth from theorem \ref{thm:dalelane} is random. The kind of inequality in the the theorem is called \textbf{oracle inequality}, as it is not possible to obtain the values on each side. They involve the unknown $f_X\pa{x}$. The estimation of errors in cross-validation kernel estimation is hard, but in practice it often works well.
\end{remark}

\section{Sobolev class and symmetric kernel}
\subsection{Review of Fourier transform}
\begin{definition}
	The characteristic function of a random variable $X$ is
	\begin{equation*}
		\varphi_X\pa{w} = \E\bra{e^{iwX}} = \int_\R e^{iwx} f_X\pa{x}dx.
	\end{equation*}
\end{definition}

\begin{remark}\label{rem:fourier_l2}
	It is possible as well to define the Fourier transform of $f\in L^2\pa{\R}$. Therefore, we take a sequence $f_m \in L^1\pa{\R}\cap L^2\pa{\R}$ such that $\norm{f-f_m}^2_2\rightarrow 0$ as $m\rightarrow \infty$ and define the Fourier transform of $f$ as the $L^2$-limit of $\F \bra{f_m}$. More precisely, we may take $f_m\pa{x} = f\pa{x} \one_{\abs{x}\leq m}$. It is in $L^2$ as the product of an $L^2\pa{\R}$ function and a bounded function, and it is in $L^1\pa{\R}$ as a result of the Cauchy-Schwarz inequality:
	\begin{equation*}
		\int_{\R} f\pa{x} \one_{\abs{x}\leq m} dx \leq \sqrt{\int_{\R} f\pa{x}^2 dx } \sqrt{\int_{-m}^m 1 dx } = \sqrt{2m}  \sqrt{\underbrace{\int_{\R} f\pa{x}^2 dx}_{<\infty} }.
	\end{equation*}
	Moreover,
	\begin{equation}\label{eq:cauchy}
		\norm{f_m -f}^2_2 = \int_{-\infty}^m \abs{f\pa{x}}^2 dx + \int_m^\infty \abs{f\pa{x}}^2 dx \rightarrow 0
	\end{equation}
	as $m\rightarrow \infty$.
	% \begin{exercise}
	% 	If $f$ is symmetric, $\F \bra{f}$ is real-valued.
	% \end{exercise}
	By equation \ref{eq:cauchy}, for all $m,m', \norm{f_m -f_{m'}}_2^2\rightarrow
		0$ as $m,m'\rightarrow \infty$, i.e.~$\pa{f_m}$ is a Cauchy sequence. By
	Plancherel's theorem \ref{prop:plancherel_inverse},
	\begin{equation*}
		\norm{\F \bra{f_m} -\F\bra{f_{m'}} }_2^2 = \norm{\F \bra{f_m -f_{m'} }}_2^2 = 2\pi \norm{f_m -f_{m'}}_2^2.
	\end{equation*}
	Thus, $\F\bra{f_m}$ is a Cauchy sequence in $L^2\pa{\R}$, so that it admits a limit in $L^2\pa{\R}$, since $L^2\pa{\R}$ is a complete normed space. We can then define the Fourier transform of $f$ to be this limit.
\end{remark}
% \begin{exercise}
% 	Prove
% 	\begin{enumerate}
% 		\item $\F\bra{f\pa{\cdot}} \pa{w} = a \F \bra{f\pa{\cdot}}\pa{w}$,
% 		\item $\F\bra{\frac{1}{h}f\pa{\frac{\pa{\cdot}}{h}}}\pa{w} = \F\bra{f\pa{\cdot}}\pa{hw}$,
% 		\item $\F\bra{f\pa{t-\cdot}}\pa{w} = e^{iwt} \F\bra{f\pa{\cdot}} \pa{-w}$.
% 	\end{enumerate}
% \end{exercise}
% \begin{exercise}
% 	If $\hat{f}_X$ is a kernel density estimator, where the kernel is symmetric, then
% 	\begin{enumerate}
% 		\item $\F\bra{K}$ is symmetric and real,
% 		\item $\F\bra{\hat{f}_X}\pa{w} = \phi_u \pa{w} \F\bra{K} \pa{hw}$,
% 	\end{enumerate}
% 	where $\phi_u \pa{w} = \frac{1}{n}\sum_{j=1}^n e^{iwX_j}$.
% \end{exercise}
% \begin{exercise}
% 	By 2 of \ref{prop:plancherel_inverse}, check that the Fourier transform of the sinc kernel is $\one_{\abs{w}\leq 1}$.
% \end{exercise}

% Let $f\in L^1\pa{\R}$ such that $\F\bra{f}\in L^1\pa{\R}\cap L^2\pa{\R}$. Then
% \begin{equation*}
% 	f\pa{x} = \frac{1}{2\pi} \int_{\R} e^{-iwx}\F\bra{f}\pa{w} d w.
% \end{equation*}

% We can define, after differentiating in the usual way without knowing whether the right-hand side is differentiable,
% \begin{equation*}
% 	f'\pa{x} \defeq \frac{i}{2\pi} \int_{\R} \pa{-iw} e^{-iwx} \F\bra{f}\pa{w}dw,
% \end{equation*}
% and
% \begin{equation*}
% 	f''\pa{x} \defeq \frac{1}{2\pi} \int_\R \pa{w^2} e^{-iwx} \F\bra{f} \pa{w}dw,
% \end{equation*}
% and so on. 
We notice that the characteristic function of a random variable is the Fourier
transform of its density. Therefore, the density function $f_X$ of a random
variable $X$ is the inverse Fourier transform of its characteristic function
$\varphi_X$.
\begin{equation*}
	f_X\pa{x} = \frac{1}{2\pi}^d \int_{\R^d} e^{-iwx} \varphi_X\pa{w}dw.
\end{equation*}
We can get the derivative of the density function by differentiating the Fourier transform of the density function.
\begin{equation*}
	f_X^{(m)}\pa{x}= \frac{1}{2\pi^d} \int_{\R^d} \pa{-iw}^m e^{-iwx} \varphi_X\pa{w}dw.
\end{equation*}
It means that $\F\bra{f_X^{(m)}}\pa{w} = \pa{-iw}^m \varphi_X\pa{w}.$

\subsection{Sobolev class}
Building on this, we make the following definition.
\begin{definition}[Sobolev class]\label{def:sobolev_class}
	Let $\beta>0$, $L>0$, the Sobolev class $\calp_S\pa{\beta,L}$ is defined as
	\begin{equation*}
		\calp_S\pa{\beta,L} = \set{f: f \text{ is a density on }  \R \text{ and } \int_{\R}\abs{w}^{2\beta} \abs{\F\bra{f}\pa{w}}^2 dw \leq 2\pi L^2}.
	\end{equation*}
\end{definition}
The restriction is basically saying that the $L^2$ norm of the function $f_X^{(m)}$ is bounded by $L$. This is a generalization of the $L^2$ norm of the function $f_X\pa{x}$, which is the $L^2$ norm of the density function $f$.
